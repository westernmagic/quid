\documentclass[a4paper, landscape]{article}

\usepackage{geometry}
\geometry{
	includeheadfoot,
	left   = 1cm,
	right  = 1cm,
	top    = 1cm,
	bottom = 1cm
}
\usepackage{amsmath}

\usepackage{scrlayer-scrpage}
\usepackage{ccicons}

\lohead*{Quantum Rosetta}
\cohead*{}
\rohead*{}
\lofoot*{\ccbysa\\Michal Sudwoj \& QuID Team}
\cofoot*{\thepage}
\rofoot*{QuID 2018\\v1.1, 2018-09-12}

\usepackage{fontspec}
\usepackage{luaotfload}
\setmainfont{Kannada_MN.ttc}
\setmonofont{Source Code Pro}
% \setmathrm{Latin Modern}

\usepackage{xcolor}
\usepackage{tabu}
\usepackage{colortbl}
\usepackage{makecell}

\usepackage{hyperref}
\usepackage{biblatex}
\addbibresource{Quantum_Computing.bib}
\nocite{*}

\usepackage{multicol}
\usepackage{minted}
\setminted{
	autogobble,
	python3,
	linenos,
	tabsize = 2,
	xleftmargin = 0.75cm
}

\usepackage{ifthen}
\provideboolean{grayscale}

% Wikipedia table colors
\ifthenelse{\boolean{grayscale}}{
	\usemintedstyle{bw}
	\def\yes{}
	\def\part{\cellcolor[gray]{0.8}}
	\def\no{\cellcolor[gray]{0.6}{\normalfont \textemdash}}
}{
	\def\yes{\cellcolor[RGB]{153, 255, 153}}
	\def\part{\cellcolor[RGB]{255, 255, 187}}
	\def\no{\cellcolor[RGB]{255, 153, 153}}
}

\begin{document}
{\Large Builtins and Primitives}\\
\begin{tabu}{
		X[-1]
		|| >{\texttt\bgroup} X[-1] <{\egroup}
		|  >{\texttt\bgroup} X[-1] <{\egroup}
		|  >{\texttt\bgroup} X[-1] <{\egroup}
		|  >{\texttt\bgroup} X[-1] <{\egroup}
		|  >{\texttt\bgroup} X[-1] <{\egroup}
	} \rowfont{\normalfont}
	Gate                      & Q\#                     & ProjectQ      & Cirq                & Qiskit       & PyQuil       \\ \hline
	\part $I$                 & \yes  I                 & \no           & \no                 & \yes iden    & \yes I       \\
	\yes  $H$                 & \yes  H                 & \yes  H       & \yes H              & \yes h       & \yes H       \\
	\yes  $S$                 & \yes  S                 & \yes  S       & \yes S              & \yes s       & \yes S       \\
	\yes  $T$                 & \yes  T                 & \yes  T       & \yes T              & \yes t       & \yes T       \\
	\yes  $X$, NOT            & \yes  X                 & \yes  X       & \yes X              & \yes x       & \yes X       \\
	\yes  $Y$                 & \yes  Y                 & \yes  Y       & \yes Y              & \yes y       & \yes Y       \\
	\yes  $Z$                 & \yes  Z                 & \yes  Z       & \yes Z              & \yes z       & \yes Z       \\
	\yes  $R_x$               & \yes  Rx                & \yes  Rx      & \yes RotXGate       & \yes rx      & \yes Rx      \\
	\yes  $R_y$               & \yes  Ry                & \yes  Ry      & \yes RotYGate       & \yes ry      & \yes Ry      \\
	\yes  $R_z$               & \yes  Rz                & \yes  Rz      & \yes RotZGate       & \yes rz      & \yes Rz      \\
	\yes  $R_\phi$            & \yes  R1                & \yes  R       & \no                 & \yes u\_1    & \yes PHASE   \\ \hline \hline
	\yes  Measure             & \yes  M                 & \yes  Measure & \yes measure        & \yes measure & \yes MEASURE \\
	\part Barrier             & \no                     & \yes  Barrier & \no                 & \yes barrier & \no          \\ \hline \hline
	\yes  CX, CNOT            & \yes  CNOT              & \yes  CNOT    & \yes CNOT           & \yes cx      & \yes CNOT    \\
	\yes  CCX, CCNOT, Toffoli & \yes  CCNOT             & \yes  Toffoli & \yes CCX, TOFFOLI   & \yes ccx     & \yes CCNOT   \\
	\yes  SWAP                & \yes  SWAP              & \yes  Swap    & \yes SwapGate       & \yes swap    & \yes SWAP    \\
	\yes  CZ                  & \part (Controlled Z)    & \yes  CZ      & \yes CZ             & \yes cz      & \yes CZ      \\
	\yes  CSWAP, Fredkin      & \part (Controlled SWAP) & \part C(Swap) & \yes CSWAP, FREDKIN & \yes cswap   & \yes CSWAP   \\
	\part CR\textsubscript{z} & \part (Controlled Rz)   & \yes  CRz     & \no                 & \yes crz     & \yes CPHASE  \\
	\part ISWAP               & \no                     & \no           & \yes ISWAP          & \no          & \yes ISWAP   \\
	\part QFT                 & \yes  QFT               & \yes  QFT     & \no                 & \no          & \no          \\ \hline \hline
	Other
		& HY, RAll0, RAll1
		& Sdag, Tdag, SqrtX, SqrtSwap, Entangle, TimeEvolution, QubitOperator, PhaseOracle, PermutationOracle, AddConstantModN
		& Rot11Gate, CCZ
		& cy, ch, rzz
		& PSWAP, CPHASE00, CPHASE01, CPHASE10
	\\
\end{tabu}
\vfill
\printbibliography[heading = none]
\newpage

{\Large Features} \\
\begin{tabu}{
		X[-1]
		|| >{\texttt\bgroup} X[-1] <{\egroup}
		|  >{\texttt\bgroup} X[-1] <{\egroup}
		|  >{\texttt\bgroup} X[-1] <{\egroup}
		|  >{\texttt\bgroup} X[-1] <{\egroup}
		|  >{\texttt\bgroup} X[-1] <{\egroup}
} \rowfont{\normalfont}
	Operation
		& Q\#
		& ProjectQ
		& Cirq
		& Qiskit
		& PyQuil
	\\ \hline
	\makecell[l]{Gate from\\matrix}
		& \no  %TODO
		& \yes \makecell[l]{
			G\ \ \ \ \ \ \ \ = BasicGate()\\
			G.matrix = numpy.matrix(A)
		}
		& \yes SingleQubitMatrixGate, TwoQubitMatrixGate
		& \no
		& \yes defgate
	\\ \hline
	Controlled
		& \yes (Controlled G)(c, q)
		& \yes \makecell[l]{
			C(G) | q\ \ \ \ \ \ \ \# or\\
			with Control(eng, c):\\
			\ \ \ \ G | q
		}
		& \yes ControlledGate(G)
		& \yes G.q\_if(q)
		& \no  %TODO
	\\ \hline
	Inverse
		& \yes (Adjoint G)(q)
		& \yes  \makecell[l]{
			with Dagger(eng):\\
			\ \ \ \ G | q
		}
		& \yes G.inverse()
		& \yes G.inverse()
		& \yes Program(G(q)).dagger()
	\\ \hline
	% $\sqrt{}$
	% 	& \no
	% 	& \no
	% 	& \yes  G ** 0.5
	% 	& \no
	% 	& \no
	% \\ \hline
	\makecell[l]{Apply to\\many qubits}
	& \yes ApplyToEach(G, qs)
		& \yes \makecell[l]{
			All(G)\ \ \ \ | qs \# or\\
			Tensor(G) | qs
		}
		& \yes G.on\_each(qs)
		& \yes G(qs)
		& \no  % TODO
	\\ \hline
	\rowfont{\normalfont}
	Simlators
		& \yes  local
		& \yes  local \& cloud
		& \yes  local
		& \yes  local \& cloud
		& \part cloud only
	\\ \hline
	\makecell[l]{Execute on real\\quantum computer}
		& \no  no
		& \yes IBM
		& \yes Google
		& \yes IBM
		& \yes Rigetti
	\\ \hline
	\rowfont{\normalfont}
	Rotation units
		& radians
		& radians
		& \textbf{half-turns}, radians, degrees
		& radians
		& radians
	\\ \hline
	\rowfont{\normalfont}
	Integrations
		&
		& Fermilib, OpenFermion
		& OpenFermion, OpenQAsm
		& Qiskit-Aqua, OpenQAsm
		& OpenFermion
	\\
\end{tabu}

\vspace{0.5cm}
%{\Large Code examples} \\
\begin{tabu*}{ X[-1] | X[-1] }
	% TODO: write custom highlighter
	\begin{minted}{csharp}
		// Q#
		// Circuit.qs
		namespace Circuit {
		  open Microsoft.Quantum.Primitive;
		  open Microsoft.Quantum.Canon;

		  operation Circuit() : (Int) {
		    body {
		      mutable result = 0;
		      using (q = Qubit[1]) {
		        H(q[0]);
		        if (M(q[0]) == One) {
		          set result = 1;
		        }
		        Set(Zero, q[0]);
		      }
		      return result;
		    }
		    // adjoint            auto;
		    // controlled         auto;
		    // controlled adjoint auto;
		  }
		}
	\end{minted}
	&
	\begin{minted}{csharp}
		// Q#
		// Driver.cs
		using Microsoft.Quantum.Simulation.Core;
		using Microsoft.Quantum.Simulation.Simulators;

		namespace Circuit {
		  class Driver {
		    static void Main(string[] args) {
		      using (var sim = new QuantumSimulator()) {
		        var result = Circuit.Run(sim).Result;
		        System.Console.WriteLine($"Measured: {result}");
		      }
		    }
		  }
		}
	\end{minted}
\end{tabu*}

\begin{minted}{xml}
	<!-- MyProgram.csproj -->
	<Project Sdk="Microsoft.NET.Sdk">
	  <PropertyGroup>
	    <OutputType>Exe</OutputType>
	    <TargetFramework>netcoreapp2.1</TargetFramework>
	    <RootNamespace>MyProgram</RootNamespace>
	  </PropertyGroup>

	  <ItemGroup>
	    <PackageReference Include="Microsoft.Quantum.Canon"           Version="0.2.1809.701-preview" />
	    <PackageReference Include="Microsoft.Quantum.Development.Kit" Version="0.2.1809.701-preview" />
	  </ItemGroup>
	</Project>
\end{minted}

\newpage

\begin{tabu*}{ X[1.25] | X[1] }
	\begin{minted}{python}
		# ProjectQ
		from projectq     import MainEngine
		from projectq.ops import *


		eng = MainEngine()
		q   = eng.allocate_qubit()


		H       | q
		Measure | q

		eng.flush()


		print("Measured: {}".format(
		    int(q)
		))
	\end{minted}
	&
	\begin{minted}{python}
		# Cirq
		from cirq import *



		q       = GridQubit(0, 0)
		circuit = Circuit()


		circuit.append([H(q)])
		circuit.append([measure(q, key = "c")])

		sim    = google.XmonSimulator()
		result = sim.run(circuit)

		print("Measured: {}".format(
		    int(result.measurements["c"][0,0])
		))
	\end{minted}
	\\ \hline
	\begin{minted}{python}
		# Qiskit
		from qiskit import *



		q       = QuantumRegister(1)
		c       = ClassicalRegister(1)
		circuit = QuantumCircuit(q, c)

		circuit.h(q)
		circuit.measure(q, c)

		sim    = execute(circuit, "local_qasm_simulator", shots = 1)
		result = sim.result()

		print("Measured: {}".format(
		    list(result.get_counts())[0]
		))
	\end{minted}
	&
	\begin{minted}{python}
		# PyQuil
		from pyquil.quil  import Program
		from pyquil.gates import *
		from pyquil.api   import QVMConnection

		program = Program()



		program.inst(H(0))
		program.inst(MEASURE(0, 0))

		qvm    = QVMConnection()
		result = qvm.run(program, [0])

		print(result)
	\end{minted}
	\\
\end{tabu*}

\newpage

\end{document}
